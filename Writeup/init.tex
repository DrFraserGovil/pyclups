	\section{Initialisation}

			In the case where $\vec{w}$ must be optimised, we must first choose an initial starting point, $\vec{w}_0$, for the starting optimisation. The closer this is to the optimal point, the quicker the optimiser will converge.

			In order to make this efficient, we rewrite the constraint vector $\vec{c}$ in the following fashion:
			\begin{equation}
				\vec{c}(\vec{w}) = \vec{\xi} + \psi(\vec{w})
			\end{equation}
			Here $\vec{\xi}$ contains both the equality constraints and any constant-offsets associated with the inequality constraints, such that we may then enforce the following conditions on $\vec{\psi}$:
			\begin{equation}
				\left[ \vec{\psi}(\vec{w}) \right]_i  \begin{cases}
						= 0 & \text{if $i$ exact constraint}
						\\
						\geq 0 & \text{else}
					\end{cases}
			\end{equation}
			For example, if condition $j$ is that $x_j \geq -4$, then $\xi_j = -4$ and $\psi_j = \exp(w_j)$. We also limit ourselves to the case where $\vec{w} = \psi^{-1}(\vec{c} - \vec{\xi})$ exists. Note that this is not a limitation on our general method, but rather a choice made for efficient initialisation.

			The algorithm for determining the initialisation point is then:
			\begin{enumerate}
				\item Compute $\{ \vec{a}^\text{BLUP} \}$ and hence $\hat{\vec{Z}}^\text{blup}$: the predictors using the normal BLUP algorithm
				\item Let $\tilde{\vec{c}} = B \vec{p}_t = \vec{\xi} + \tilde{\vec{\varphi}}$
				\item Project onto the constraint-meeting surface:
				$$ \varphi_j = \begin{cases} \tilde{\varphi}_j & \text{if } \tilde{\varphi}_j \geq 0
					\\
					0 & \text{else} \end{cases}$$
				\item Set $\vec{w}_0 = \psi^{-1}\left(\varphi_j\right)$
			\end{enumerate}
			In practice, a small amount of numerical tolerance might be required (setting a $\varphi_j =0$ when $\psi^{-1} = \ln(\varphi)$ is not numerically stable), so at step 3 we suggest setting $\varphi_j = \epsilon$, some very small numerical quantity.

			\subsection{Comments on Initialisation}

				This method of initialisation is a na\"ive projection from the BLUP onto the space of constraint-obeying functions. In some simple cases, this projection is in fact equal to the global maximum: the case of positive functions, for example - the na\"ive projection truncates the BLUP to be equal to 0 wherever the BLUP would become negative, which is exactly the global solution. 
				
				In some pathological cases, however, this projection might lead to a function extremely far away from both the global maximum and the BLUP: consider the case of a BSCLUP constrained to be monotonically increasing, but where the BLUP is monotonically \textit{decreasing}. In this case, the projection of $\vec{w}_0$ would result in a flat line at the height of $Z^{BLUP}_0$ -- a rather significant deviation, and unlikely to be close to the optimum.

				Experimentally, we find that this initialisation serves as a good initial \textit{ansatz} as to the location of the optimum point for most real-world applications.